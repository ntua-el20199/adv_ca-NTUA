\documentclass{article}
\usepackage{fontspec}  % For custom font support (required by xelatex)
\usepackage[greek,english]{babel}
\setmainfont{Times New Roman}  % You can use any system-installed font

\usepackage{graphicx}
\usepackage{listings}
\usepackage{xcolor}

\lstdefinestyle{cppstyle}{
    language=C++,
    basicstyle=\ttfamily\footnotesize,
    keywordstyle=\color{blue},
    commentstyle=\color{gray},
    stringstyle=\color{teal},
    backgroundcolor=\color{white},
    frame=single,
    breaklines=true,
    breakatwhitespace=false,
    tabsize=4,
    showstringspaces=false,
    captionpos=b
}

\title{Simulating Branch Prediction with PIN}
\author{Χαράλαμπος Παπαδόπουλος \\03120199}
\date{Απρίλιος 2025}

\begin{document}

\maketitle

\section{Εισαγωγή}

Ζητούμενο της άσκησης είναι η κατασκευή και σε συνέχεια η σύγκριση διάφορων branch predictors χρησιμοποιώντας το εργαλείο προσομοίωσης PIN.

\section{Ανάλυση εντολών άλματος}
Αρχικά, συλλέγουμε δεδομένα σχετικά με τα benchmarks που θα χρησιμοποιήσουμε (τόσο για τα train όσο και για τα ref) χρησιμοποιώντας το pintool cslab\_branch\_stats.so. Λαμβάνουμε, λοιπόν τα εξής αποτελέσματα:

\begin{table}[h!]
    \centering
    \begin{tabular}{|c|c|c|c|c|c|c|c|}
        \hline
        Benchmark & Total Instructions & Total Branches & Taken & NotTaken & Uncond-Branches & Calls & Returns \\
        \hline
        401.bzip2 & 296379957910 & 48731240680 & 14828890244 & 29233486500 & 4503343406 & 82760267 & 82760263 \\
        \hline
        mcf & 1000000 & 300000 & 200000 & 100000 \\
        sjeng & 1000000 & 250000 & 175000 & 75000 \\
        xalancbmk & 1000000 & 400000 & 300000 & 100000 \\
        \hline
    \end{tabular}
    \caption{Στατιστικά εντολών άλματος για τα benchmarks}
\end{table}

\end{document}
